% -----
% TODOs
% -----

\newcommand{\jpkcomment}[1]{\todo[color=red!80,size=\scriptsize,fancyline,author=Joost-Pieter]{#1}\xspace}
\newcommand{\jpkcommentinline}[1]{\todo[inline,color=red!80,author=Joost-Pieter]{#1}\xspace}
\newcommand{\atjpk}{\colorbox{red!80}{\textbf{@Joost-Pieter}}}

\newcommand{\pscomment}[1]{\todo[color=blue!25,size=\scriptsize,fancyline,author=Philipp]{#1}\xspace}
\newcommand{\pscommentinline}[1]{\todo[inline,color=blue!25,author=Philipp]{#1}\xspace}
\newcommand{\atps}{\colorbox{blue!25}{\textbf{@Philipp}}}

\newcommand{\dhcomment}[1]{\todo[color=cyan!25,size=\scriptsize,fancyline,author=Darion]{#1}\xspace}
\newcommand{\dhcommentinline}[1]{\todo[inline,color=cyan!25,author=Darion]{#1}\xspace}
\newcommand{\atdh}{\colorbox{cyan!25}{\textbf{@Darion}}}

% -------
% Spacing
% -------

\newcommand{\morespace}[1]{~{}#1{}~}
\newcommand{\qmorespace}[1]{\quad{}#1{}\quad}
\newcommand{\qqmorespace}[1]{\qquad{}#1{}\qquad}
\newcommand{\rqmorespace}[1]{\quad{}#1{}}
\newcommand{\rqqmorespace}[1]{\qquad{}#1{}}
\newcommand{\lqmorespace}[1]{{}#1{}\quad}
\newcommand{\lqqmorespace}[1]{{}#1{}\qquad}
\newcommand{\ppreceq}{\morespace{\preceq}}
\newcommand{\ssucceq}{\morespace{\succeq}}
\newcommand{\eeq}{\morespace{=}}
\newcommand{\qeq}{\qmorespace{=}}
\newcommand{\qqeq}{\qqmorespace{=}}
\newcommand{\mmodels}{\morespace{\models}}
\newcommand{\qmodels}{\qmorespace{\models}}
\newcommand{\qqmodels}{\qqmorespace{\models}}
\newcommand{\rrightarrow}{\morespace{\rightarrow}}
\newcommand{\lleftarrow}{\morespace{\leftarrow}}
\newcommand{\lleq}{\morespace{\leq}}
\newcommand{\ggeq}{\morespace{\geq}}
\newcommand{\eequiv}{\morespace{\equiv}}
\newcommand{\hheyloleq}{\morespace{\heyloleq}}
\newcommand{\hheylogeq}{\morespace{\heylogeq}}
\newcommand{\eexpleq}{\morespace{\expleq}}
\newcommand{\eexpgeq}{\morespace{\expgeq}}

\newcommand{\qiff}{\qmorespace{\textnormal{iff}}}
\newcommand{\qqiff}{\qqmorespace{\textnormal{iff}}}
\newcommand{\lqiff}{\lqmorespace{\textnormal{iff}}}
\newcommand{\lqqiff}{\lqqmorespace{\textnormal{iff}}}

\newcommand{\qimplies}{\qmorespace{\textnormal{implies}}}
\newcommand{\qqimplies}{\qqmorespace{\textnormal{implies}}}
\newcommand{\lqimplies}{\lqmorespace{\textnormal{implies}}}
\newcommand{\lqqimplies}{\lqqmorespace{\textnormal{implies}}}

\newcommand{\qand}{\qmorespace{\textnormal{and}}}
\newcommand{\qqand}{\qqmorespace{\textnormal{and}}}
\newcommand{\rqand}{\qmorespace{\textnormal{and}}}
\newcommand{\rqqand}{\qqmorespace{\textnormal{and}}}

\newcommand{\qor}{\qmorespace{\textnormal{or}}}
\newcommand{\qqor}{\qqmorespace{\textnormal{or}}}
\newcommand{\rqor}{\qmorespace{\textnormal{or}}}
\newcommand{\rqqor}{\qqmorespace{\textnormal{or}}}

% ------
% Colors
% ------

\definecolor{hintgray}{RGB}{92,92,92}

\definecolor{col1}{RGB}{254,195,10}
\definecolor{col2}{RGB}{251,0,6}
\definecolor{col3}{RGB}{252,11,135}
\definecolor{col4}{RGB}{17,139,2}
\definecolor{col5}{RGB}{14,131,254}
\definecolor{col6}{RGB}{82,0,135}

\definecolor{brLightGreen}{HTML}{addd8e}
\definecolor{brLightGray}{HTML}{f0f0f0}

\definecolor{orange}{RGB}{230,159,0}
\definecolor{skyblue}{RGB}{86,180,233}
\definecolor{brBlue}{RGB}{0,114,178}
\definecolor{bluishgreen}{RGB}{0,158,115}
\definecolor{vermillion}{RGB}{213,94,0}
\definecolor{reddishpurple}{RGB}{204,121,167}

\colorlet{heyvlColor}{vermillion!60!black}
\definecolor{prepostColor}{RGB}{0,69,107}
\colorlet{stmtColor}{bluishgreen!50!black}

\newcommand{\boxSpec}[1]{{\colorbox{prepostColor}{\color{white}{$#1$}}}}
\newcommand{\boxInvariant}[1]{{\colorbox{heyvlColor}{\color{white}{$#1$}}}}

% -----------
% other stuff
% -----------
\newcommand{\gutter}[1]{\text{\color{gray}\footnotesize{}{#1}}\quad}
\newcommand{\lineNumber}[1]{\text{\color{gray}\footnotesize{}{(l.~#1)}}}

\newcommand{\highlightCode}[2]{\hspace{-\fboxsep}\colorbox{#1}{#2}}

% -----
% Logic
% -----

\newcommand{\Sup}{\reflectbox{\textnormal{\textsf{\fontfamily{phv}\selectfont S}}}\hspace{.2ex}}
\newcommand{\Inf}{\raisebox{.6\depth}{\rotatebox{-30}{\textnormal{\textsf{\fontfamily{phv}\selectfont \reflectbox{J}}}}\hspace{-.1ex}}}

\newcommand{\entails}{\models}

\newcommand{\quant}[2]{#1.~#2}
\newcommand{\squant}[2]{\ensuremath{\Sup \quant{#1}{#2}}}
\newcommand{\iquant}[2]{\ensuremath{\Inf \quant{#1}{#2}}}

\newcommand{\true}{\mathsf{true}}
\newcommand{\false}{\mathsf{false}}

\newcommand{\impl}{\rightarrow}
\newcommand{\coimpl}{\leftsquigarrow}
\newcommand{\coneg}{{\sim}}

\newcommand{\hla}{\ensuremath{\varphi}}
\newcommand{\hlb}{\ensuremath{\psi}}
\newcommand{\hlc}{\ensuremath{\rho}}
\newcommand{\hld}{\ensuremath{\gamma}}

\newcommand{\colheylo}[1]{{\color{prepostColor}#1}}
\newcommand{\colheyvl}[1]{{\color{heyvlColor}#1}}
\newcommand{\hhla}{\colheylo{\hla}}
\newcommand{\hhlb}{\colheylo{\hlb}}
\newcommand{\hhlc}{\colheylo{\hlc}}
\newcommand{\hhld}{\colheylo{\hld}}

\newcommand{\expa}{\colheylo{\ensuremath{X}}}
\newcommand{\expb}{\colheylo{\ensuremath{Y}}}
\newcommand{\expc}{\colheylo{\ensuremath{Z}}}

\newcommand{\expaP}{\colheylo{\ensuremath{X'}}}
\newcommand{\expbP}{\colheylo{\ensuremath{Y'}}}

\newcommand{\expleq}{\ensuremath{\preceq}}
\newcommand{\expgeq}{\ensuremath{\succeq}}

\newcommand{\expvalidate}[1]{\ensuremath{\triangle\!\left( #1 \right)}}
\newcommand{\expcovalidate}[1]{\ensuremath{\triangledown\!\left( #1 \right)}}

\newcommand{\substBy}[2]{[{#1} \mapsto {#2}]}

\newcommand{\iverson}[1]{\left[{#1}\right]}

\newcommand{\exptransId}{\ensuremath{\mathsf{id}}}

\newcommand{\exptransa}{\ensuremath{\Psi}}
\newcommand{\exptransb}{\ensuremath{\Xi}}
\newcommand{\exptransc}{\ensuremath{\Phi}}

\newcommand{\exptransleq}{\ensuremath{\mathrel{\dot{\preceq}}}}
\newcommand{\exptransgeq}{\ensuremath{\mathrel{\dot{\succeq}}}}

\newcommand{\ite}[3]{\mathrm{ite}({#1}, {#2}, {#3})}

\newcommand{\lam}[2]{\lambda {#1}.~{#2}}

\newcommand{\ifThenElse}[3]{
	\begin{cases}{#2}, & \text{if } {#1}\\{#3}, & \text{otherwise}
	\end{cases}
}
\newcommand{\ifThenElseDot}[3]{
	\begin{cases}{#2}, & \text{if } {#1}\\{#3}, & \text{otherwise}~.
	\end{cases}
}

% -------------
% Domain Theory
% -------------

\DeclareMathOperator{\opMin}{~\sqcap~}
\DeclareMathOperator{\opMax}{~\sqcup~}
% two available notations below:
%\newcommand{\binMin}[2]{\min \Set{ {#1},~ {#2}} }
%\newcommand{\binMax}[2]{\max \Set{ {#1},~ {#2}} }
\newcommand{\binMin}[2]{{#1} \opMin {#2}}
\newcommand{\binMax}[2]{{#1} \opMax {#2}}
\newcommand{\lfp}[2]{\mathrm{lfp}\,{#1}.~{#2}}
\newcommand{\gfp}[2]{\mathrm{gfp}\,{#1}.~{#2}}

\newcommand{\semlfp}[2]{\interpret{\mathrm{lfp}}\,{#1}.~{#2}}
\newcommand{\semgfp}[2]{\interpret{\mathrm{gfp}}\,{#1}.~{#2}}

% --------------
% Program Syntax
% --------------

\newcommand{\pexp}[2]{#1 \cdot \langle #2 \rangle }
\newcommand{\pexpand}{+}
% names
\newcommand{\sfsymbol}[1]{\textsf{\upshape {#1}}}
\newcommand{\Vars}{\sfsymbol{Vars}\xspace}
\newcommand{\Type}[1]{\sfsymbol{type}({#1})}
\newcommand{\typeof}[2]{\ensuremath{#1 \colon {#2}}}
\newcommand{\Types}{\mathsf{Types}\xspace}
\newcommand{\pGCL}{\sfsymbol{pGCL}\xspace}
\newcommand{\HeyVL}{\sfsymbol{HeyVL}\xspace}
\newcommand{\HeyLo}{\sfsymbol{HeyLo}\xspace}
\newcommand{\ArithExp}{\sfsymbol{AExp}\xspace}
\newcommand{\BExp}{\sfsymbol{BExp}\xspace}

%\newcommand{\typefont}[1]{\sfsymbol{#1}}
\newcommand{\funcfont}[1]{\sfsymbol{#1}}
\newcommand{\listtype}{\typefont{Lists}}
\newcommand{\listlen}{\funcfont{len}}

\newcommand{\typevar}{\ensuremath{\tau}}

\newcommand{\heyloleq}{\sqsubseteq}
\newcommand{\heylogeq}{\sqsupseteq}

\newcommand{\aexpr}{\ensuremath{a}} %arithmetic expression
\newcommand{\expr}{\ensuremath{e}} % "normal" expressions
\newcommand{\bexpr}{\ensuremath{b}} %Boolean expression
\newcommand{\procname}{P} % procedure name

\newcommand{\ufuncs}{\ensuremath{\mathcal{F}}}
\newcommand{\funcsymb}{\ensuremath{f}}
\newcommand{\termvar}{\ensuremath{t}}

\newcommand{\Bools}{\mathbb{B}}
\newcommand{\Nats}{\mathbb{N}}
\newcommand{\Ints}{\mathbb{Z}}
\newcommand{\Rats}{\mathbb{Q}}
\newcommand{\PosRats}{\mathbb{Q}_{\geq 0}}
\newcommand{\Reals}{\mathbb{R}}
\newcommand{\PosReals}{\mathbb{R}_{\geq 0}}
\newcommand{\PosRealsInf}{\mathbb{R}_{\geq 0}^{\infty}}

\newcommand{\States}{\mathsf{States}}
\newcommand{\State}{\sigma}
\newcommand{\Vals}{\mathsf{Vals}}
\newcommand{\Mod}[1]{\mathrm{Mod}({#1})}

% symbols
\newcommand{\symStmt}[1]{\ensuremath{{\color{stmtColor}{#1}}}}
\newcommand{\symSkip}{\symStmt{\texttt{skip}}}
\newcommand{\symAssign}{\coloneqq}
\newcommand{\symRasgn}{\colonapprox}
\newcommand{\symSemi}{\symStmt{\texttt{;}~}}
\newcommand{\seq}{\symSemi} % \seq is a shorter alias for \symSemi
\newcommand{\symIf}{\symStmt{\texttt{if}}}
\newcommand{\symElse}{\symStmt{\texttt{else}}}
\newcommand{\symAssert}{\symStmt{\texttt{assert}}}
\newcommand{\symAssume}{\symStmt{\texttt{assume}}}
\newcommand{\symNondet}{\symStmt{\symIf~(\ast)}}
\newcommand{\symDemonic}{\symStmt{\symIf~(\sqcap)}}
\newcommand{\symCdot}{\symStmt{\symIf~(\cdot)}}
\newcommand{\symAngelic}{\symStmt{\symIf~(\sqcup)}}
\newcommand{\symScale}{\symStmt{\texttt{scale}}}
\newcommand{\symHavoc}{\symStmt{\texttt{havoc}}}
\newcommand{\symWhile}{\symStmt{\texttt{while}}}
\newcommand{\symNegate}{\symStmt{\texttt{negate}}}
\newcommand{\symObserve}{\symStmt{\texttt{observe}}}
\newcommand{\symTick}{\symStmt{\texttt{reward}}}
\newcommand{\symProc}{\symStmt{\texttt{proc}}}
\newcommand{\symcoProc}{\symStmt{\texttt{coproc}}}
\newcommand{\symReturns}{\texttt{->}}
\newcommand{\symPre}{\symStmt{\texttt{pre}}}
\newcommand{\symPost}{\symStmt{\texttt{post}}}
\newcommand{\symValidate}{\symStmt{\texttt{validate}}}
\newcommand{\symVar}{\symStmt{\texttt{var}}}
\newcommand{\symAnnotate}{\symStmt{\texttt{@}}}
\newcommand{\symLoop}{\texttt{loop}}
\newcommand{\symBreak}{\texttt{break}}

\newcommand{\blockStart}{\ensuremath{\{}}
\newcommand{\blockEnd}{\ensuremath{\}}}
\newcommand{\midElse}{\blockEnd~\symElse~\blockStart~}

% procedures

%\newcommand{\proc}[3]{\symProc~{\mathit{#1}}\,\texttt{({#2})}~\symReturns~\texttt{({#3})}}
\newcommand{\proc}[3]{\symProc~{\mathit{#1}}\,\texttt{(}{#2}\texttt{)}~\symReturns~\texttt{(}{#3}\texttt{)}}
\newcommand{\coproc}[3]{\symcoProc~{\mathit{#1}}\,\texttt{(}{#2}\texttt{)}~\symReturns~\texttt{(}{#3}\texttt{)}}
\newcommand{\Ensures}[1]{\symPost~{\color{prepostColor}#1}}
\newcommand{\Requires}[1]{\symPre~{\color{prepostColor}#1}}

\newcommand{\varandtype}[2]{\ensuremath{{#1}\texttt{:}\,\mathsf{#2}}}
\newcommand{\varin}{\ensuremath{\mathit{in}}}
\newcommand{\varout}{\ensuremath{\mathit{out}}}
\newcommand{\multi}[1]{\ensuremath{\overline{#1}}}

% statements
\newcommand{\stmt}{{\color{heyvlColor}\ensuremath{S}}}
\newcommand{\stmtP}{{\color{heyvlColor}\ensuremath{S'}}}
\newcommand{\stmtC}{{\color{heyvlColor}\ensuremath{S_C}}}
\newcommand{\sstmt}{{\color{heyvlColor}\stmt}}
\newcommand{\stmtOne}{{\color{heyvlColor}\ensuremath{S_1}}}
\newcommand{\stmtOneP}{{\color{heyvlColor}\ensuremath{S_1'}}}
\newcommand{\stmtTwo}{{\color{heyvlColor}\ensuremath{S_2}}}
\newcommand{\stmtTwoP}{{\color{heyvlColor}\ensuremath{S_2'}}}
\newcommand{\stmtN}[1]{{\color{heyvlColor}\ensuremath{S_{#1}}}}
\newcommand{\stmtI}{{\stmtN{i}}}
\newcommand{\stmtITransformed}{{\color{heyvlColor}\ensuremath{\tilde{\stmtI}}}}
\newcommand{\stmtTransformed}{{\color{heyvlColor}\ensuremath{\tilde{\stmt}}}}

\newcommand{\pGCLstmt}{\ensuremath{P}}

\newcommand{\stmtSkip}{\symSkip}
\newcommand{\stmtAsgn}[2]{\ensuremath{{#1} \symAssign {#2}}}
\newcommand{\stmtDecl}[2]{\symVar~\ensuremath{{#1}\colon {#2}}}
\newcommand{\stmtDeclInit}[3]{\symVar~\ensuremath{{#1}\colon {#2} \symRasgn {#3}}}
\newcommand{\stmtDeclInitDirac}[3]{\symVar~\ensuremath{{#1}\colon {#2} = {#3}}}
\newcommand{\stmtRasgn}[2]{\ensuremath{{#1} \symRasgn {#2}}}
\newcommand{\stmtSeq}[2]{\ensuremath{{#1}\symSemi {#2}}}
\newcommand{\stmtIf}[3]{\ensuremath{\symIf~({#1})~\{ {#2} \}~\symElse~\{ {#3} \}}}
\newcommand{\stmtIfStart}[1]{\ensuremath{\symIf~({#1})~\{ }}
\newcommand{\stmtProb}[3]{\ensuremath{\{ {#2} \} ~[{#1}]~ \{ {#3} \}}}
\newcommand{\stmtWhile}[2]{\ensuremath{\symWhile~({#1})~\{ {#2} \}}}
\newcommand{\headerWhile}[1]{\ensuremath{\symWhile~({#1})}}
\newcommand{\stmtWhileInv}[3]{\ensuremath{\symWhile~({#1})~\texttt{invariant}~{#2}~\{ {#3} \}}}
\newcommand{\stmtHavoc}[1]{\ensuremath{\symHavoc~{#1}}}
\newcommand{\stmtAssert}[1]{\ensuremath{\symAssert~{#1}}}
\newcommand{\stmtAssume}[1]{\ensuremath{\symAssume~{#1}}}
\newcommand{\stmtNondet}[2]{\ensuremath{\symNondet~\{ {#1} \}~\symElse~\{ {#2} \}}}
\newcommand{\stmtSpec}[3]{{#1} : [{#2},~ {#3}]}
\newcommand{\stmtLoop}[1]{\ensuremath{\symLoop~\{ {#1} \}}}
\newcommand{\stmtBreak}{\symBreak}

\newcommand{\stmtAnnotate}[3]{\ensuremath{\symAnnotate{}{\symStmt{\texttt{#1}}}({#2})~{#3}}}

\newcommand{\stmtDemonicStart}{\ensuremath{\symDemonic~\{ }}
\newcommand{\stmtCdotStart}{\ensuremath{\symCdot~\{ }}
\newcommand{\stmtElseStart}{\ensuremath{\symElse~\{ }}
\newcommand{\stmtAngelicStart}{\ensuremath{\symAngelic~\{ }}

\newcommand{\stmtDemonic}[2]{\ensuremath{\stmtDemonicStart{#1} \}~\stmtElseStart {#2} \}}}
\newcommand{\stmtCdot}[2]{\ensuremath{\stmtCdotStart{#1} \}~\stmtElseStart {#2} \}}}

\newcommand{\stmtAngelic}[2]{\ensuremath{\stmtAngelicStart{#1} \}~\stmtElseStart {#2} \}}}
\newcommand{\stmtScale}[1]{\ensuremath{\symScale~{#1}}}
\newcommand{\stmtNegate}{\ensuremath{\symNegate}}
\newcommand{\stmtObserve}[1]{\ensuremath{\symObserve~{#1}}}
\newcommand{\stmtTick}[1]{\ensuremath{\symTick~{#1}}}
\newcommand{\stmtValidate}{\ensuremath{\symValidate}}
\newcommand{\stmtVar}[2]{\ensuremath{\symVar~\typeof{#1}{#2}}}

% down versions of statements
\newcommand{\Havoc}[1]{\ensuremath{\symHavoc~{#1}}}
\newcommand{\Assert}[1]{\ensuremath{\symAssert~{#1}}}
\newcommand{\Assume}[1]{\ensuremath{\symAssume~{#1}}}
\newcommand{\Compare}[1]{\ensuremath{\symCompare~{#1}}}
\newcommand{\Nondet}[2]{\ensuremath{\symDemonic~\{ {#1} \}~\symElse~\{ {#2} \}}}
\newcommand{\Negate}{\ensuremath{\stmtNegate}}
\newcommand{\Validate}{\ensuremath{\stmtValidate}}

% up versions of statements
\newcommand{\symUp}{\symStmt{\texttt{co}}}

\newcommand{\upHavoc}[1]{\ensuremath{\symUp\symHavoc~{#1}}}
\newcommand{\upAssert}[1]{\ensuremath{\symUp\symAssert~{#1}}}
\newcommand{\upAssume}[1]{\ensuremath{\symUp\symAssume~{#1}}}
\newcommand{\upCompare}[1]{\ensuremath{\symUp\symCompare~{#1}}}
\newcommand{\upNondet}[2]{\ensuremath{\symAngelic~\{ {#1} \}~\symElse~\{ {#2} \}}}
\newcommand{\upNegate}{\ensuremath{\symUp\stmtNegate}}
\newcommand{\upValidate}{\ensuremath{\symUp\stmtValidate}}

\newcommand{\coHavoc}[1]{\ensuremath{\symUp\symHavoc~{#1}}}
\newcommand{\coAssert}[1]{\ensuremath{\symUp\symAssert~{#1}}}
\newcommand{\coAssume}[1]{\ensuremath{\symUp\symAssume~{#1}}}
\newcommand{\coCompare}[1]{\ensuremath{\symUp\symCompare~{#1}}}
\newcommand{\coNondet}[2]{\ensuremath{\symAngelic~\{ {#1} \}~\symElse~\{ {#2} \}}}
\newcommand{\coNegate}{\ensuremath{\symUp\stmtNegate}}
\newcommand{\coValidate}{\ensuremath{\symUp\stmtValidate}}

% expressions
\newcommand{\exprTrue}{\ensuremath{\texttt{true}}}
\newcommand{\exprFalse}{\ensuremath{\texttt{false}}}
\newcommand{\exprFlip}[1]{\ensuremath{\symStmt{\texttt{flip}}({#1})}}
\newcommand{\exprUnif}[2]{\ensuremath{\symStmt{\texttt{unif}}({#1},~{#2})}}

\newcommand{\interpretState}[1]{\interpret{#1}(\State)}
\newcommand{\interpretsimpleState}[1]{\interpretsimple{#1}(\State)}
\newcommand{\interpretStateSubstBy}[3]{\interpret{#1}(\State\substBy{#2}{#3})}
\newcommand{\interpretsimpleStateSubstBy}[3]{\interpretsimple{#1}(\State\substBy{#2}{#3})}

\newcommand{\eval}[1]{\llbracket{#1}\rrbracket}
\newcommand{\evalState}[1]{\eval{#1}(\State)}
\newcommand{\evalStateSubstBy}[3]{\eval{#1}(\State\substBy{#2}{#3})} % better name

\newcommand{\exprIte}[3]{\mathrm{ite}({#1},~{#2},~{#3})}
\newcommand{\exprLet}[3]{\mathrm{let}({#1},~{#2},~{#3})}

% ---------------------------------------------
% Weakest Pre-condition/Pre-expectation Calculi
% ---------------------------------------------

\newcommand{\symWp}{\sfsymbol{wp}}
\newcommand{\symVc}{\sfsymbol{vp}}
\newcommand{\symVcB}{\sfsymbol{vc}}

% \wp is already defined as the "Weierstrass elliptic function": https://tex.stackexchange.com/questions/476660/what-is-the-name-of-the-character-wp
% just overwrite it
\renewcommand{\wp}[1]{\symWp\llbracket{#1}\rrbracket}

\newcommand{\vc}[1]{\symVc\llbracket{#1}\rrbracket}
\newcommand{\vcB}[1]{\symVcB\llbracket{#1}\rrbracket}

\newcommand{\symWlp}{\sfsymbol{wlp}}
\newcommand{\wlp}[1]{\symWlp\llbracket{#1}\rrbracket}

\newcommand{\symCwp}{\sfsymbol{cwp}}
\newcommand{\cwp}[1]{\symCwp\llbracket{#1}\rrbracket}

\newcommand{\symErt}{\sfsymbol{ert}}
\newcommand{\ert}[1]{\symErt\llbracket{#1}\rrbracket}

\newcommand{\Expectations}{\mathbb{E}}
\newcommand{\OneBoundedExpectations}{\Expectations_{\leq 1}}

\newcommand{\ber}[1]{\ensuremath{\mathsf{ber!}({#1})}}

\newcommand{\embed}[1]{\scalebox{0.85}{\textnormal{\textsf{?}}}({#1})}
\newcommand{\coembed}[1]{\scalebox{0.85}{\textnormal{\textsf{$\neg$?}}}({#1})}

\newcommand{\reach}[2]{\mathrm{reach}({#1},~ {#2})}

\newcommand{\wpPhi}{\prescript{\symWp}{}{\Phi}}
\newcommand{\wlpPhi}{\prescript{\symWlp}{}{\Phi}}

\newcommand{\lowerTriple}[4][]{\langle {\colheylo{#2}} \rangle_{\expleq}^{#1}~{#3}~\langle {\colheylo{#4}} \rangle}
\newcommand{\upperTriple}[4][]{\langle {\colheylo{#2}} \rangle_{\expgeq}^{#1}~{#3}~\langle {\colheylo{#4}} \rangle}

\newcommand{\lowerVcTriple}[3]{\lowerTriple[\symVc]{#1}{#2}{#3}}
\newcommand{\upperVcTriple}[3]{\upperTriple[\symVc]{#1}{#2}{#3}}

\newcommand{\lowerWpTriple}[3]{\lowerTriple[\symWp]{#1}{#2}{#3}}
\newcommand{\upperWpTriple}[3]{\upperTriple[\symWp]{#1}{#2}{#3}}

\newcommand{\lowerWlpTriple}[3]{\lowerTriple[\symWlp]{#1}{#2}{#3}}
\newcommand{\upperWlpTriple}[3]{\upperTriple[\symWlp]{#1}{#2}{#3}}

\newcommand{\intersem}[1]{\color{hintgray}{//~#1}}

\newcommand{\exphavoc}[2]{{#1} \ominus {#2}}

% ---------------------------------------------
% Slicing
% ---------------------------------------------

\newcommand{\substmt}{\ensuremath{\sqsubseteq}}

\newcommand{\slice}{\ensuremath{\stmtP}}

\newcommand*\circled[1]{\tikz[baseline=(char.base)]{
		\node[shape=circle,draw,inner sep=0.5pt] (char) {#1};}}

\newcommand{\stmtNo}[1]{\textnormal{\circled{#1}}}

\newcommand{\varEnabled}[1]{enabled_{#1}}

\newcommand{\ParkIndCondPre}{\ensuremath{\mathsf{PI}_{pre}}}
\newcommand{\ParkIndCondInd}{\ensuremath{\mathsf{PI}_{ind}}}
\newcommand{\ParkIndCondPost}{\ensuremath{\mathsf{PI}_{post}}}

\newcommand{\sliced}[1]{\text{\sout{\transparent{0.5}{\ensuremath{#1}}}}}


% Scatter plots
% Based on TeX Source from
% Accurately Computing Expected Visiting Times and Stationary Distributions in Markov Chains
% Hannah Mertens, Joost-Pieter Katoen, Tim Quatmann, Tobias Winkler
% https://doi.org/10.48550/arXiv.2401.10638 (v2)

\newcommand{\sliceMethod}[1]{\texttt{#1}\xspace}
\newlength{\scatterplotsize}
\setlength{\scatterplotsize}{6cm}
\newlength{\numberruntimeplotheight}
\setlength{\numberruntimeplotheight}{6cm}

\definecolor{barplotcolora}{HTML}{1b9e77}
\definecolor{barplotcolorb}{HTML}{d95f02}
\definecolor{barplotcolorc}{HTML}{7570b3}
\definecolor{barplotcolord}{HTML}{e7298a}

\tikzset{slicingMethod/.code={%% only marks, possible
			% erroring slicing
			\ifthenelse{\equal{#1}{first_cex}}{\tikzset{red, only marks, mark=x, mark size=1.5pt, line width=1pt}}{}%, mark=*, }}{}% mark size=1.5pt
			\ifthenelse{\equal{#1}{optimal_cex}}{\tikzset{blue, only marks, mark=+, mark size=1.5pt, line width=1pt}}{}%, mark=*, }}{}% mark size=1.5pt

			% verifying slicing
			\ifthenelse{\equal{#1}{core}}{\tikzset{red, only marks, mark=*, mark size=1.5pt, line width=1pt}}{}%, mark=*, }}{}% mark size=1.5pt
			\ifthenelse{\equal{#1}{mus}}{\tikzset{blue, only marks, mark=+, mark size=1.5pt, line width=1pt}}{}%, mark=*, }}{}% mark size=1.5pt
			\ifthenelse{\equal{#1}{sus}}{\tikzset{green, only marks, mark=x, mark size=1.5pt, line width=1pt}}{}%, mark=*, }}{}% mark size=1.5pt
			\ifthenelse{\equal{#1}{exists_forall}}{\tikzset{orange, only marks, mark=o, mark size=1.5pt, line width=1pt}}{}%, mark=*, }}{}% mark size=1.5pt
		}}


\tikzset{slicingMethodBar/.code={%% only marks, possible
			% erroring slicing
			\ifthenelse{\equal{#1}{first_cex}}{\tikzset{red, only marks, mark=x, mark size=1.5pt, line width=1pt}}{}%, mark=*, }}{}% mark size=1.5pt
			\ifthenelse{\equal{#1}{optimal_cex}}{\tikzset{blue, only marks, mark=+, mark size=1.5pt, line width=1pt}}{}%, mark=*, }}{}% mark size=1.5pt

			% verifying slicing
			\ifthenelse{\equal{#1}{core}}{\tikzset{barplotcolora,fill=barplotcolora}}{}%, mark=*, }}{}% mark size=1.5pt
			\ifthenelse{\equal{#1}{mus}}{\tikzset{barplotcolorb,fill=barplotcolorb}}{}%, mark=*, }}{}% mark size=1.5pt
			\ifthenelse{\equal{#1}{sus}}{\tikzset{barplotcolorc,fill=barplotcolorc}}{}%, mark=*, }}{}% mark size=1.5pt
			\ifthenelse{\equal{#1}{exists_forall}}{\tikzset{barplotcolord,fill=barplotcolord}}{}%, mark=*, }}{}% mark size=1.5pt
		}}
\newcommand{\numberruntimeplot}[7]{%
	\begin{tikzpicture}
		\begin{axis}[
				width=\scatterplotsize,
				height=\numberruntimeplotheight,
				ymin=0,
				ymax=#7,
				axis y line=left,
				ytick={0,5,10,15,20},
				extra y ticks = {25},
				extra y tick labels = {OOR},
				xmin=1,
				xmax=80000,
				xtick={10,100,1000},
				axis x line=bottom,
				xmode=log,
				extra x ticks = {1, 30000},
				extra x tick labels = {$\leq 1$, OOR},
				ymode=normal,
				%xtick=data,
				ylabel= #4,
				xlabel= #5,
				xlabel style={font=\scriptsize},
				ylabel style={yshift=-6pt, font=\scriptsize},
				xticklabel style={font=\scriptsize},
				yticklabel style={font=\scriptsize},
				legend columns=\legendcols,
				legend style={\legendstyle, font=\scriptsize,
						nodes={scale=1, transform shape},inner sep=2pt},
				%every axis plot/.append style={ultra thick}, for presentation
				legend cell align={left},
				table/col sep=comma,
			]
			\foreach \slicingMethod in {#2}{%
					\edef\loopbody{
						\noexpand\addplot[slicingMethod=\slicingMethod ] table [
								x expr=\noexpand\thisrow{\slicingMethod_slice_time} > 0 ? \noexpand\thisrow{\slicingMethod_slice_time} : 1,
								y expr=\noexpand\thisrow{\slicingMethod_slice_size} > 10000 ? 25 : \noexpand\thisrow{\slicingMethod_slice_size},
								col sep=comma] {#1};
					}
					\loopbody
				}
			\draw[densely dotted] (axis cs: 10,0) -- (axis cs: 10,#7);
			\draw[densely dotted] (axis cs: 100,0) -- (axis cs: 100,#7);
			\draw[densely dotted] (axis cs: 1000,0) -- (axis cs: 1000,#7);
			\draw[densely dotted] (axis cs: 30000,0) -- (axis cs: 30000,#7);
			\draw[densely dotted] (axis cs: 0,25) -- (axis cs: 30000,25);
			\legend{#3}
		\end{axis}
		%		\pgfresetboundingbox
		%		\useasboundingbox (-1,-0.85) rectangle (4.5,4.9);
	\end{tikzpicture}%
}

\newcommand{\relativenumberruntimeplot}[7],
				extra y ticks = {1.2},
				extra y tick labels = {OOR},
				xmin=1,
				xmax=80000,
				xtick={10,100,1000},
				axis x line=bottom,
				xmode=log,
				extra x ticks = {1, 30000},
				extra x tick labels = {$\leq 1$, OOR},
				ymode=normal,
				%xtick=data,
				ylabel= #4,
				xlabel= #5,
				xlabel style={font=\scriptsize},
				ylabel style={yshift=-6pt, font=\scriptsize},
				xticklabel style={font=\scriptsize},
				yticklabel style={font=\scriptsize},
				legend columns=\legendcols,
				legend style={\legendstyle, font=\scriptsize,
						nodes={scale=1, transform shape},inner sep=2pt},
				%every axis plot/.append style={ultra thick}, for presentation
				legend cell align={left},
				table/col sep=comma,
			]
			\foreach \slicingMethod in {#2}{%
					\edef\loopbody{
						\noexpand\addplot[slicingMethod=\slicingMethod ] table [
								x expr=\noexpand\thisrow{\slicingMethod_slice_time} > 0 ? \noexpand\thisrow{\slicingMethod_slice_time} : 1,
								y expr=\noexpand\thisrow{\slicingMethod_slice_size} > 10000 ? 1.2 : (\noexpand\thisrow{\slicingMethod_slice_size}/\noexpand\thisrow{original_size}),
								col sep=comma] {#1};
					}
					\loopbody
				}
			\draw[densely dotted] (axis cs: 10,0) -- (axis cs: 10,#7);
			\draw[densely dotted] (axis cs: 100,0) -- (axis cs: 100,#7);
			\draw[densely dotted] (axis cs: 1000,0) -- (axis cs: 1000,#7);
			\draw[densely dotted] (axis cs: 30000,0) -- (axis cs: 30000,#7);
			\draw[densely dotted] (axis cs: 0,1.2) -- (axis cs: 30000,1.2);
			\legend{#3}
		\end{axis}
		%		\pgfresetboundingbox
		%		\useasboundingbox (-1,-0.85) rectangle (4.5,4.9);
	\end{tikzpicture}%
}


\newcommand{\relativeslicesizeruntimeplot}[7]{%

	\begin{tikzpicture}
		\begin{groupplot}[group style = {
						group size = 1 by 2,
						vertical sep = 15},
				width = \linewidth,
				tick label style={font=\scriptsize},
				typeset ticklabels with strut,
				enlarge y limits=false,
				legend columns=\legendcols,
				legend style={\legendstyle, font=\scriptsize,
						nodes={scale=1, transform shape},inner sep=2pt},
				%every axis plot/.append style={ultra thick}, for presentation
				legend cell align={left},
			]
			\nextgroupplot[
				xmin=0, xmax=21,
				ymin=0, ymax=80000,
				height = 0.31\linewidth,
				ymode = log,
				axis y line* = left,
				grid=both,
				ybar = .04cm,
				bar width = 1.5pt,
				enlarge x limits = {value = 1},
				xticklabel=\empty,
				ytick={0.1,10,100, 1000, 10000, 30000},
				yticklabels={0,10, $10^2$, $10^3$, $10^4$, TO},
				xtick distance=1,
				%xtick = {1000, 10000}, yticklabels = {,,}
			]

			\def\slicingMethod{core}
			\edef\loopbody{
				\noexpand\addplot[slicingMethodBar=\slicingMethod ] table [
						x expr=\noexpand\thisrow{id},
						y expr=\noexpand\thisrow{\slicingMethod_slice_time} > 0 ? \noexpand\thisrow{\slicingMethod_slice_time} : 1,
						% if sus failed => don't show in plot by using negative value (-10)
						% if sus terminated, but method not => method OOR (1.2)
						% if sus, method terminated, sus sliced nothing => method is perfect (0)
						%y expr=\noexpand\thisrow{sus_slice_size} > 10000 ? -10 : ( \noexpand\thisrow{\slicingMethod_slice_size} > 10000 ? 1.2 : ((\noexpand\thisrow{original_size}-\noexpand\thisrow{sus_slice_size} < 1) ? 0 : ((\noexpand\thisrow{original_size}-\noexpand\thisrow{\slicingMethod_slice_size})/(\noexpand\thisrow{original_size}-\noexpand\thisrow{sus_slice_size})))),
						col sep=comma] {#1};
			}
			\loopbody

			\def\slicingMethod{mus}
			\edef\loopbody{
				\noexpand\addplot[slicingMethodBar=\slicingMethod ] table [
						x expr=\noexpand\thisrow{id},
						y expr=\noexpand\thisrow{\slicingMethod_slice_time} > 0 ? \noexpand\thisrow{\slicingMethod_slice_time} : 1,
						% if sus failed => don't show in plot by using negative value (-10)
						% if sus terminated, but method not => method OOR (1.2)
						% if sus, method terminated, sus sliced nothing => method is perfect (0)
						%y expr=\noexpand\thisrow{sus_slice_size} > 10000 ? -10 : ( \noexpand\thisrow{\slicingMethod_slice_size} > 10000 ? 1.2 : ((\noexpand\thisrow{original_size}-\noexpand\thisrow{sus_slice_size} < 1) ? 0 : ((\noexpand\thisrow{original_size}-\noexpand\thisrow{\slicingMethod_slice_size})/(\noexpand\thisrow{original_size}-\noexpand\thisrow{sus_slice_size})))),
						col sep=comma] {#1};
			}
			\loopbody

			\def\slicingMethod{sus}
			\edef\loopbody{
				\noexpand\addplot[slicingMethodBar=\slicingMethod ] table [
						x expr=\noexpand\thisrow{id},
						y expr=\noexpand\thisrow{\slicingMethod_slice_time} > 0 ? \noexpand\thisrow{\slicingMethod_slice_time} : 1,
						% if sus failed => don't show in plot by using negative value (-10)
						% if sus terminated, but method not => method OOR (1.2)
						% if sus, method terminated, sus sliced nothing => method is perfect (0)
						%y expr=\noexpand\thisrow{sus_slice_size} > 10000 ? -10 : ( \noexpand\thisrow{\slicingMethod_slice_size} > 10000 ? 1.2 : ((\noexpand\thisrow{original_size}-\noexpand\thisrow{sus_slice_size} < 1) ? 0 : ((\noexpand\thisrow{original_size}-\noexpand\thisrow{\slicingMethod_slice_size})/(\noexpand\thisrow{original_size}-\noexpand\thisrow{sus_slice_size})))),
						col sep=comma] {#1};
			}
			\loopbody

			\def\slicingMethod{exists_forall}
			\edef\loopbody{
				\noexpand\addplot[slicingMethodBar=\slicingMethod ] table [
						x expr=\noexpand\thisrow{id},
						y expr=\noexpand\thisrow{\slicingMethod_slice_time} > 0 ? \noexpand\thisrow{\slicingMethod_slice_time} : 1,
						% if sus failed => don't show in plot by using negative value (-10)
						% if sus terminated, but method not => method OOR (1.2)
						% if sus, method terminated, sus sliced nothing => method is perfect (0)
						%y expr=\noexpand\thisrow{sus_slice_size} > 10000 ? -10 : ( \noexpand\thisrow{\slicingMethod_slice_size} > 10000 ? 1.2 : ((\noexpand\thisrow{original_size}-\noexpand\thisrow{sus_slice_size} < 1) ? 0 : ((\noexpand\thisrow{original_size}-\noexpand\thisrow{\slicingMethod_slice_size})/(\noexpand\thisrow{original_size}-\noexpand\thisrow{sus_slice_size})))),
						col sep=comma] {#1};
			}
			\loopbody


			\nextgroupplot[
				xmin=0, xmax=21,
				ymin=-1.3, ymax=0,
				width = \linewidth, height = 0.25\linewidth,
				axis y line* = left,
				grid=both,
				ybar = .04cm,
				bar width = 1.5pt,
				enlarge x limits = {value = 1},
				ytick={0, -0.5, -1, -1.2},
				yticklabels={0\%, 50\%, 100\%, TO},
				xticklabel style={rotate=-45, anchor=west, font=\scriptsize\ttfamily},
				xtick distance=1,
				xticklabels={0,,navarro20\_page31,
						gehr18\_1,
						navarro20\_ex4\_3,
						navarro20\_ex4\_5,
						navarro20\_ex5\_8,
						navarro20\_figure7,
						navarro20\_figure8,
						2drwalk,
						bayesian\_network,
						prspeed,
						rdspeed,
						rdwalk,
						sprdwalk,
						unif\_gen4\_wlp,
						burglar\_alarm\_wp,
						park\_1,
						park\_5,
						park\_6,
						algorithm\_r,
						bn\_passified,}
			]

			\def\slicingMethod{core}
			\edef\loopbody{
				\noexpand\addplot[slicingMethodBar=\slicingMethod ] table [
						x expr=\noexpand\thisrow{id},
						% if sus failed => don't show in plot by using negative value (10)
						% if sus terminated, but method not => method OOR (-1.2)
						% if sus, method terminated, sus sliced nothing => method is perfect (0)
						% y expr=\noexpand\thisrow{sus_slice_size} > 10000 ? 10 : ( \noexpand\thisrow{\slicingMethod_slice_size} > 10000 ? -1.2 : ((\noexpand\thisrow{original_size}-\noexpand\thisrow{sus_slice_size} < 1) ? 0 : (-(\noexpand\thisrow{original_size}-\noexpand\thisrow{\slicingMethod_slice_size})/(\noexpand\thisrow{original_size}-\noexpand\thisrow{sus_slice_size})))),
						y expr=( \noexpand\thisrow{\slicingMethod_slice_size} > 10000 ? -1.2 : ((-(\noexpand\thisrow{\slicingMethod_slice_size})/(\noexpand\thisrow{original_size})))),
						col sep=comma] {#1};
			}
			\loopbody

			\def\slicingMethod{mus}
			\edef\loopbody{
				\noexpand\addplot[slicingMethodBar=\slicingMethod ] table [
						x expr=\noexpand\thisrow{id},
						% if sus failed => don't show in plot by using negative value (10)
						% if sus terminated, but method not => method OOR (-1.2)
						% if sus, method terminated, sus sliced nothing => method is perfect (0)
						y expr=( \noexpand\thisrow{\slicingMethod_slice_size} > 10000 ? -1.2 : ((-(\noexpand\thisrow{\slicingMethod_slice_size})/(\noexpand\thisrow{original_size})))),
						col sep=comma] {#1};
			}
			\loopbody
			\def\slicingMethod{sus}
			\edef\loopbody{
				\noexpand\addplot[slicingMethodBar=\slicingMethod ] table [
						x expr=\noexpand\thisrow{id},
						% if sus failed => don't show in plot by using negative value (10)
						% if sus terminated, but method not => method OOR (-1.2)
						% if sus, method terminated, sus sliced nothing => method is perfect (0)
						y expr=( \noexpand\thisrow{\slicingMethod_slice_size} > 10000 ? -1.2 : ((-(\noexpand\thisrow{\slicingMethod_slice_size})/(\noexpand\thisrow{original_size})))),
						col sep=comma] {#1};
			}
			\loopbody
			\def\slicingMethod{exists_forall}
			\edef\loopbody{
				\noexpand\addplot[slicingMethodBar=\slicingMethod ] table [
						x expr=\noexpand\thisrow{id},
						% if sus failed => don't show in plot by using negative value (10)
						% if sus terminated, but method not => method OOR (-1.2)
						% if sus, method terminated, sus sliced nothing => method is perfect (0)
						y expr=( \noexpand\thisrow{\slicingMethod_slice_size} > 10000 ? -1.2 : ((-(\noexpand\thisrow{\slicingMethod_slice_size})/(\noexpand\thisrow{original_size})))),
						col sep=comma] {#1};
			}
			\loopbody

		\end{groupplot}
	\end{tikzpicture}
}

\newcommand{\relativeslicesizeruntimeplotscatter}[7],
				extra y ticks = {1.2},
				extra y tick labels = {OOR},
				xmin=1,
				xmax=80000,
				xtick={10,100,1000},
				axis x line=bottom,
				xmode=log,
				extra x ticks = {1, 30000},
				extra x tick labels = {$\leq 1$, OOR},
				ymode=normal,
				%xtick=data,
				ylabel= #4,
				xlabel= #5,
				xlabel style={font=\scriptsize},
				ylabel style={yshift=-6pt, font=\scriptsize},
				xticklabel style={font=\scriptsize},
				yticklabel style={font=\scriptsize},
				legend columns=\legendcols,
				legend style={\legendstyle, font=\scriptsize,
						nodes={scale=1, transform shape},inner sep=2pt},
				%every axis plot/.append style={ultra thick}, for presentation
				legend cell align={left},
				table/col sep=comma,
			]
			\foreach \slicingMethod in {#2}{%
					\edef\loopbody{
						\noexpand\addplot[slicingMethod=\slicingMethod ] table [
								x expr=\noexpand\thisrow{\slicingMethod_slice_time} > 0 ? \noexpand\thisrow{\slicingMethod_slice_time} : 1,
								% if sus failed => don't show in plot by using negative value (-10)
								% if sus terminated, but method not => method OOR (1.2)
								% if sus, method terminated, sus sliced nothing => method is perfect (0)
								y expr=\noexpand\thisrow{sus_slice_size} > 10000 ? -10 : ( \noexpand\thisrow{\slicingMethod_slice_size} > 10000 ? 1.2 : ((\noexpand\thisrow{original_size}-\noexpand\thisrow{sus_slice_size} < 1) ? 0 : ((\noexpand\thisrow{original_size}-\noexpand\thisrow{\slicingMethod_slice_size})/(\noexpand\thisrow{original_size}-\noexpand\thisrow{sus_slice_size})))),
								col sep=comma] {#1};
					}
					\loopbody
				}
			\draw[densely dotted] (axis cs: 10,0) -- (axis cs: 10,#7);
			\draw[densely dotted] (axis cs: 100,0) -- (axis cs: 100,#7);
			\draw[densely dotted] (axis cs: 1000,0) -- (axis cs: 1000,#7);
			\draw[densely dotted] (axis cs: 30000,0) -- (axis cs: 30000,#7);
			\draw[densely dotted] (axis cs: 0,1.2) -- (axis cs: 30000,1.2);
			\legend{#3}
		\end{axis}
		%		\pgfresetboundingbox
		%		\useasboundingbox (-1,-0.85) rectangle (4.5,4.9);
	\end{tikzpicture}%
}


\newcommand{\methodcomparisonscatterplot}[5]{%
	\begin{tikzpicture}
		% \path[use as bounding box,draw=red] (-1,-1.1) rectangle (2.9,3.4);

		\begin{axis}[
				width=\scatterplotsize,
				height=\scatterplotsize,
				axis equal image,
				xmin=1,
				ymin=1,
				ymax=30000,
				xmax=30000,
				xmode=log,
				ymode=log,
				axis x line=bottom,
				axis y line=left,
				xtick={10,100,1000},
				ytick={10,100,1000},
				%xticklabels={0.1,1,10,100,1000},
				extra x ticks = {1, 6000, 20000}, %20000
				extra x tick labels = {$\leq 1$, OOR, INC}, %n/a
				extra y ticks = {1, 6000, 20000}, %20000
				extra y tick labels = {$\leq 1$, OOR, INC}, %n/a
				% TO_VALUE = -4000 # timeout
				%extra x tick style = {grid = major},
				%ytick={0.1,1,10,100,1000},
				%yticklabels={0.1,1,10,100,1000},
				%extra y ticks = {-4000,-7000,-12000},
				%extra y tick labels = {\raisebox{4000ex}{OOR/TO},NA/NS,INC/UN},
				%extra y tick style = {grid = major},
				xlabel={#3},
				xlabel style={yshift=12pt,font=\scriptsize},
				xticklabel style={font=\scriptsize,rotate=290},
				ylabel={#5},
				ylabel style={yshift=-8pt, font=\scriptsize},
				yticklabel style={font=\scriptsize},
				%legend pos=outer north east,
				legend columns=\legendcols,
				legend style={\legendstyle, font=\scriptsize,
						nodes={scale=1, transform shape},inner sep=2pt},
			]
			\addplot[
				scatter,
				only marks,
				red,
				mark=x,
				mark size=1.5pt,
				%scatter src=explicit symbolic,
				% not working:
				% visualization depends on={transient-states}\as\tsm,
				% scatter/@pre marker code/.append style={/tikz/mark size=\tsm
			]%
			table [col sep=comma,x=#2,y=#4] {#1};
			\addplot[no marks] coordinates {(0.01,0.01) (6000,6000) };
			\addplot[no marks, densely dotted] coordinates {(0.1,1) (600,6000)};
			\addplot[no marks, densely dotted] coordinates {(1,0.1) (6000,600)};
			\draw[densely dotted] (axis cs: 0.1,6000) -- (axis cs: 20000,6000);
			\draw[densely dotted] (axis cs: 0.1,20000) -- (axis cs: 20000,20000);
			\draw[densely dotted] (axis cs: 6000,0.1) -- (axis cs: 6000, 20000);
			\draw[densely dotted] (axis cs: 20000, 0.1) -- (axis cs: 20000,20000);
		\end{axis}
		\pgfresetboundingbox
		\useasboundingbox (-1,-1.0) rectangle (4.4,3.5);
	\end{tikzpicture}
}